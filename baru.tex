Raw text/code from overleaf file:
\documentclass[conference]{IEEEtran}
\IEEEoverridecommandlockouts
% The preceding line only needs to identify funding in the first footnote. If that is unneeded, please comment it out.
\usepackage{cite}
\usepackage{amsmath,amssymb,amsfonts}
\usepackage{algorithmic}
\usepackage{graphicx}
\usepackage{textcomp}
\usepackage{xcolor}
\def\BibTeX{{\rm B\kern-.05em{\sc i\kern-.025em b}\kern-.08em
    T\kern-.1667em\lower.7ex\hbox{E}\kern-.125emX}}
\begin{document}


\title{Demonstrating a Method to Create a Cheap Electrophoresis Rig Solution\\}


\author{
\IEEEauthorblockN{Authors: Srikar Baru \ \ \ \ Vikram Choudhury \ \ \ \ Pooja Thaker \ \ \ \ Nathan Martin \ \ \ \ Danyal Ahmad}
}


\maketitle


\begin{abstract}
We observed and analyzed the potential and electric field in a saltwater solution with conducting wires that could be used for a low-cost electrophoresis rig. Specifically, we inspected the formation of equipotential regions in the saltwater solution because they are especially important for the functions of an electrophoresis rig in organizing particles by charge and mass. To do this, we measured the potential difference (compared to a common point) at various set points in the saltwater solution, for various voltages. We expected to have equipotential regions that are able to be represented with elliptic curves and to have a non-constant electric field. We found that solution is a good low-cost substitute for electrophoresis rig solution as it creates equipotential regions suitable for the electrophoresis rig’s applications.
\end{abstract}


\section{Introduction(Background Information and Theory):}
Different particles will settle at different regions in a solution based on their size and charge (negative charged particles will rest at higher potential regions and positive charged particles will rest at lower potential regions). An electrophoresis rig separates particles based on size and charge by having equipotential regions that different particles will settle at. A solution that is very conductive is needed to create a good separation between different particles. Table salt (NaCl) breaks down into its ions (Na+ and Cl-) in water, allowing for a high conductivity as the ions are free to carry charge in the solution. This allows for a higher current density by the equation $J = \sigma E$, where J is current density, $\sigma$ is conductivity of the material (very high for saltwater), and E is the electric field, which allows for clearer voltage differences across the rig as a higher current means a higher voltage by Ohm’s law.  




Placing a fixed positive charge (The positive terminal of a power supply) and a fixed negative charge(The ground terminal of a power supply) into this saltwater will create an electric field. This field will not be constant(it will be maximized close to each charge and minimized at the point exactly between them) and will never be zero between the two charges (both charges' electric fields are never in opposition in the region between them). Since $V=\int Edx$, there will be a potential gradient (since E is never 0) and this potential gradient will change inconsistently (as E is not constant in the solution). 


Certain points in this gradient will have the same potential, these points are considered to have equipotential. A region of these points is called an equipotential region and will run perpendicular to the net electric field. 


\ 


\ 


\ 


In this case, they should follow this pattern:


    \includegraphics[width=.9\linewidth]{fig1.JPG}




For our purposes, we can measure the potential difference from a set point to a number of points in a region to check if that area is an equipotential region(or line).


\section{Methods and Materials:}


To carry out this experiment, the following materials were utilized: a 5 feet by 8 feet whiteboard, a marker of any distinct color (red was used for the following trials), two wires with clasps at the ends, 500 mL of water, a clear container with a base measuring 10 centimeters by 10 centimeters, and a pinch of salt. We set up this lab by pouring the water into the container and then mixing it in the salt to induce a higher conductivity within the water. We then attached each of the 
wires to the positive and negative ends of the power supply, ensuring that they were on opposite sides of the container. On a whiteboard, we drew a graph of dimensions 20 cm by 20 cm, aligning the container of saltwater’s corners with the vertices of the graph. We then measured the potential difference at each of the points marked by the graph that the power supply output. This was repeated with the power supply set to 3, 6, and 9 volts, testing each point at its respective voltage.  Additionally, note that points 2, 5, and 8 serve as the baseline for potential (they have neutral (0) potential) throughout the experiment. Following this, we computed the rate of change in potential along each path by dividing the difference between the electric potential at the start and end point of each path and dividing it by the length of the path, resulting in a value that is measured in volts per centimeter (V/cm).




\ 


\ 


\ 


\ 


\ 


\ 


Lab Setup:


    \includegraphics[width=1\linewidth]{fgi2.JPG}


\section{Results:}


Measured Voltage at different Power Supply Voltages:


Power Supply Voltage = 3V(Fig. 3):


    \includegraphics[width=.8\linewidth]{Screenshot 2024-11-30 at 7.07.04 PM.png}




Power Supply Voltage = 6V(Fig. 4):




    \includegraphics[width=.8\linewidth]{Screenshot 2024-11-30 at 7.00.34 PM.png}


\ 


Power Supply Voltage = 9V(Fig. 5):




    \includegraphics[width=0.8\linewidth]{figs3thru5fix.JPG}




We can find the rate of change of potential by inspecting the path between the positive and ground terminals of the power supply.


Rate of Change by increments along the center path for multiple voltages:




    \includegraphics[width=0.8\linewidth]{table1.JPG}




    \includegraphics[width=1\linewidth]{fig6.JPG}
The electric field for this saltwater can be defined by $E=kQ(\frac{0.1}{x^2}+\frac{0.1}{0.1-x^2})$


Between 0 and 0.1 m(The distance between the two terminals), the maximum values for this equation are found at x=0m and x= 0.1m(one denominator will be 0 so $E = \infty$) and the minimum value is found at 0.05m(When the denominators are maximized). Thus the tested rate of change makes sense as it follows that type of pattern, though it is odd that the rate of change seems to be much higher at the negative terminal than it is at the positive terminal. This is likely due to a combination of human error and errors with the system. The terminals moved around a bit while we worked, which may have changed the distance between certain points and the terminal, causing odd values. Additionally, the salt in the water may not have dissolved evenly and congregated closer to the negative terminal, increasing the relative electric field in that region since the water would be more conductive.


Beyond this, we are able to use our data to display the electric fields at each point and form equipotential lines:


\ 


    \includegraphics[width=0.9\linewidth]{fig7.JPG}
We can see that, with a small margin of error, the top and bottom left points for each of the different voltages are the same. These form an equipotential line which makes sense as both have the same distance to the positive and ground terminals, so they will both be affected by the terminals electric fields in the same way. This goes for the top and bottom right points and the center line between the terminals for each voltage as well. We can also see that as the left clasp gave off a positive charge and the right clasp gave off a negative charge, the electric fields at each point pointed away from the cathode and gradually started to point more toward the anode as they got closer to it, following the path of electric fields from positive to negative.


Though some of the potential values were a little different between supposed equipotential regions, this can be explained by human error. For the same reasons why the tested rate of change for potential was a little off.


\section{Discussion:}
By analyzing the data, we observed that the rate of change of potential follows what is essentially an inverse quadratic function that has approached infinity at x=Om and x=0.1m. This observation aligns with what we expected, as the electric field is the derivative of potential and this is the same form as the electric field for two-point charges. We also identified three equipotential lines located along the paths between the top and bottom right, the top and bottom left, and directly between the two charges. We had a few errors that skewed our data, but did not entirely reject our hypothesis as the areas where we did not make significant errors were extremely consistent with our expected results. These errors mainly had to do with the terminals of the power supply shifting around in the water and possibly with the salt not dissolving evenly in the water, Ideally, the experiment would be conducted with the terminals of the power supply affixed onto the container, as opposed to just being clamped down, and with the water stirred well so that the salt would be fully and evenly dissolved into it. Also a voltmeter with higher precision would be useful, since more exact data is better data. Despite the lack of optimization, this experiment shows that saltwater is a reasonable low cost solution for an electrophoresis rig as it can properly construct equipotential regions.


\section{Work by Person:}


Vikram Choudhury: Set up the lab and performed the experiment. Worked on the introduction (background information) section, did the formatting for the assignment and performed most of the revisions.


Srikar Baru: Set up the lab and charted data for the experiment. Worked on the abstract and conclusion sections and processed a lot of the data.


Pooja Thaker: Set up the lab and helped perform the experiment. Worked on the procedure section and created all of the diagrams that were used in the document.


Danyal Ahmad: Set up the lab and helped chart data for the experiment. Worked on the abstract, procedure, and conclusion and helped process some of the data.


Nathan Martin: Set up the lab and helped perform the experiment. Worked on the analysis and results sections and created graphs for that section.


\section{References:}


Libre Texts Equipotential Surfaces and Conductors:


[1] "7.6: Equipotential Surfaces and Conductors." *Physics LibreTexts*, LibreTexts, 
     1 Oct. 2024, phys.libretexts.org/Bookshelves/University Physics/University Physics 28OpenStax29/
     University Physics II - Thermodynamics Electricity and Magnetism 28OpenStax29/
     073A Electric Potential/7.063A Equipotential Surfaces and Conductors.


\ 


Tiplers Vol.2:


[2] Tipler, Paul A. *Physics for Scientists and Engineers: Vol. 2: Electricity and 
     Magnetism, Light*. Fourth Edition ed., W. H. Freeman, 15 Sept. 1998. Accessed 
     6 Nov. 2024.


\end{document}
















Raw logs from overleaf file:
This is pdfTeX, Version 3.141592653-2.6-1.40.26 (TeX Live 2024) (preloaded format=pdflatex 2024.8.9)  8 DEC 2024 21:00
entering extended mode
 \write18 enabled.
 %&-line parsing enabled.
**conference_101719.tex
(./conference_101719.tex
LaTeX2e <2023-11-01> patch level 1
L3 programming layer <2024-03-14>
(./IEEEtran.cls
Document Class: IEEEtran 2015/08/26 V1.8b by Michael Shell
-- See the "IEEEtran_HOWTO" manual for usage information.
-- http://www.michaelshell.org/tex/ieeetran/
\@IEEEtrantmpdimenA=\dimen140
\@IEEEtrantmpdimenB=\dimen141
\@IEEEtrantmpdimenC=\dimen142
\@IEEEtrantmpcountA=\count188
\@IEEEtrantmpcountB=\count189
\@IEEEtrantmpcountC=\count190
\@IEEEtrantmptoksA=\toks17
LaTeX Font Info:    Trying to load font information for OT1+ptm on input line 503.
(/usr/local/texlive/2024/texmf-dist/tex/latex/psnfss/ot1ptm.fd
File: ot1ptm.fd 2001/06/04 font definitions for OT1/ptm.
)
-- Using 8.5in x 11in (letter) paper.
-- Using PDF output.
\@IEEEnormalsizeunitybaselineskip=\dimen143
-- This is a 10 point document.
\CLASSINFOnormalsizebaselineskip=\dimen144
\CLASSINFOnormalsizeunitybaselineskip=\dimen145
\IEEEnormaljot=\dimen146
LaTeX Font Info:    Font shape `OT1/ptm/bx/n' in size <5> not available
(Font)              Font shape `OT1/ptm/b/n' tried instead on input line 1090.
LaTeX Font Info:    Font shape `OT1/ptm/bx/it' in size <5> not available
(Font)              Font shape `OT1/ptm/b/it' tried instead on input line 1090.
LaTeX Font Info:    Font shape `OT1/ptm/bx/n' in size <7> not available
(Font)              Font shape `OT1/ptm/b/n' tried instead on input line 1090.
LaTeX Font Info:    Font shape `OT1/ptm/bx/it' in size <7> not available
(Font)              Font shape `OT1/ptm/b/it' tried instead on input line 1090.
LaTeX Font Info:    Font shape `OT1/ptm/bx/n' in size <8> not available
(Font)              Font shape `OT1/ptm/b/n' tried instead on input line 1090.
LaTeX Font Info:    Font shape `OT1/ptm/bx/it' in size <8> not available
(Font)              Font shape `OT1/ptm/b/it' tried instead on input line 1090.
LaTeX Font Info:    Font shape `OT1/ptm/bx/n' in size <9> not available
(Font)              Font shape `OT1/ptm/b/n' tried instead on input line 1090.
LaTeX Font Info:    Font shape `OT1/ptm/bx/it' in size <9> not available
(Font)              Font shape `OT1/ptm/b/it' tried instead on input line 1090.
LaTeX Font Info:    Font shape `OT1/ptm/bx/n' in size <10> not available
(Font)              Font shape `OT1/ptm/b/n' tried instead on input line 1090.
LaTeX Font Info:    Font shape `OT1/ptm/bx/it' in size <10> not available
(Font)              Font shape `OT1/ptm/b/it' tried instead on input line 1090.
LaTeX Font Info:    Font shape `OT1/ptm/bx/n' in size <11> not available
(Font)              Font shape `OT1/ptm/b/n' tried instead on input line 1090.
LaTeX Font Info:    Font shape `OT1/ptm/bx/it' in size <11> not available
(Font)              Font shape `OT1/ptm/b/it' tried instead on input line 1090.
LaTeX Font Info:    Font shape `OT1/ptm/bx/n' in size <12> not available
(Font)              Font shape `OT1/ptm/b/n' tried instead on input line 1090.
LaTeX Font Info:    Font shape `OT1/ptm/bx/it' in size <12> not available
(Font)              Font shape `OT1/ptm/b/it' tried instead on input line 1090.
LaTeX Font Info:    Font shape `OT1/ptm/bx/n' in size <17> not available
(Font)              Font shape `OT1/ptm/b/n' tried instead on input line 1090.
LaTeX Font Info:    Font shape `OT1/ptm/bx/it' in size <17> not available
(Font)              Font shape `OT1/ptm/b/it' tried instead on input line 1090.
LaTeX Font Info:    Font shape `OT1/ptm/bx/n' in size <20> not available
(Font)              Font shape `OT1/ptm/b/n' tried instead on input line 1090.
LaTeX Font Info:    Font shape `OT1/ptm/bx/it' in size <20> not available
(Font)              Font shape `OT1/ptm/b/it' tried instead on input line 1090.
LaTeX Font Info:    Font shape `OT1/ptm/bx/n' in size <24> not available
(Font)              Font shape `OT1/ptm/b/n' tried instead on input line 1090.
LaTeX Font Info:    Font shape `OT1/ptm/bx/it' in size <24> not available
(Font)              Font shape `OT1/ptm/b/it' tried instead on input line 1090.
\IEEEquantizedlength=\dimen147
\IEEEquantizedlengthdiff=\dimen148
\IEEEquantizedtextheightdiff=\dimen149
\IEEEilabelindentA=\dimen150
\IEEEilabelindentB=\dimen151
\IEEEilabelindent=\dimen152
\IEEEelabelindent=\dimen153
\IEEEdlabelindent=\dimen154
\IEEElabelindent=\dimen155
\IEEEiednormlabelsep=\dimen156
\IEEEiedmathlabelsep=\dimen157
\IEEEiedtopsep=\skip48
\c@section=\count191
\c@subsection=\count192
\c@subsubsection=\count193
\c@paragraph=\count194
\c@IEEEsubequation=\count195
\abovecaptionskip=\skip49
\belowcaptionskip=\skip50
\c@figure=\count196
\c@table=\count197
\@IEEEeqnnumcols=\count198
\@IEEEeqncolcnt=\count199
\@IEEEsubeqnnumrollback=\count266
\@IEEEquantizeheightA=\dimen158
\@IEEEquantizeheightB=\dimen159
\@IEEEquantizeheightC=\dimen160
\@IEEEquantizeprevdepth=\dimen161
\@IEEEquantizemultiple=\count267
\@IEEEquantizeboxA=\box51
\@IEEEtmpitemindent=\dimen162
\IEEEPARstartletwidth=\dimen163
\c@IEEEbiography=\count268
\@IEEEtranrubishbin=\box52
)
** ATTENTION: Overriding command lockouts (line 2).
(/usr/local/texlive/2024/texmf-dist/tex/latex/cite/cite.sty
LaTeX Info: Redefining \cite on input line 302.
LaTeX Info: Redefining \nocite on input line 332.
Package: cite 2015/02/27  v 5.5
) (/usr/local/texlive/2024/texmf-dist/tex/latex/amsmath/amsmath.sty
Package: amsmath 2023/05/13 v2.17o AMS math features
\@mathmargin=\skip51
For additional information on amsmath, use the `?' option.
(/usr/local/texlive/2024/texmf-dist/tex/latex/amsmath/amstext.sty
Package: amstext 2021/08/26 v2.01 AMS text
(/usr/local/texlive/2024/texmf-dist/tex/latex/amsmath/amsgen.sty
File: amsgen.sty 1999/11/30 v2.0 generic functions
\@emptytoks=\toks18
\ex@=\dimen164
)) (/usr/local/texlive/2024/texmf-dist/tex/latex/amsmath/amsbsy.sty
Package: amsbsy 1999/11/29 v1.2d Bold Symbols
\pmbraise@=\dimen165
) (/usr/local/texlive/2024/texmf-dist/tex/latex/amsmath/amsopn.sty
Package: amsopn 2022/04/08 v2.04 operator names
)
\inf@bad=\count269
LaTeX Info: Redefining \frac on input line 234.
\uproot@=\count270
\leftroot@=\count271
LaTeX Info: Redefining \overline on input line 399.
LaTeX Info: Redefining \colon on input line 410.
\classnum@=\count272
\DOTSCASE@=\count273
LaTeX Info: Redefining \ldots on input line 496.
LaTeX Info: Redefining \dots on input line 499.
LaTeX Info: Redefining \cdots on input line 620.
\Mathstrutbox@=\box53
\strutbox@=\box54
LaTeX Info: Redefining \big on input line 722.
LaTeX Info: Redefining \Big on input line 723.
LaTeX Info: Redefining \bigg on input line 724.
LaTeX Info: Redefining \Bigg on input line 725.
\big@size=\dimen166
LaTeX Font Info:    Redeclaring font encoding OML on input line 743.
LaTeX Font Info:    Redeclaring font encoding OMS on input line 744.
\macc@depth=\count274
LaTeX Info: Redefining \bmod on input line 905.
LaTeX Info: Redefining \pmod on input line 910.
LaTeX Info: Redefining \smash on input line 940.
LaTeX Info: Redefining \relbar on input line 970.
LaTeX Info: Redefining \Relbar on input line 971.
\c@MaxMatrixCols=\count275
\dotsspace@=\muskip16
\c@parentequation=\count276
\dspbrk@lvl=\count277
\tag@help=\toks19
\row@=\count278
\column@=\count279
\maxfields@=\count280
\andhelp@=\toks20
\eqnshift@=\dimen167
\alignsep@=\dimen168
\tagshift@=\dimen169
\tagwidth@=\dimen170
\totwidth@=\dimen171
\lineht@=\dimen172
\@envbody=\toks21
\multlinegap=\skip52
\multlinetaggap=\skip53
\mathdisplay@stack=\toks22
LaTeX Info: Redefining \[ on input line 2953.
LaTeX Info: Redefining \] on input line 2954.
) (/usr/local/texlive/2024/texmf-dist/tex/latex/amsfonts/amssymb.sty
Package: amssymb 2013/01/14 v3.01 AMS font symbols
(/usr/local/texlive/2024/texmf-dist/tex/latex/amsfonts/amsfonts.sty
Package: amsfonts 2013/01/14 v3.01 Basic AMSFonts support
\symAMSa=\mathgroup4
\symAMSb=\mathgroup5
LaTeX Font Info:    Redeclaring math symbol \hbar on input line 98.
LaTeX Font Info:    Overwriting math alphabet `\mathfrak' in version `bold'
(Font)                  U/euf/m/n --> U/euf/b/n on input line 106.
)) (/usr/local/texlive/2024/texmf-dist/tex/latex/algorithms/algorithmic.sty
Package: algorithmic 2009/08/24 v0.1 Document Style `algorithmic'
(/usr/local/texlive/2024/texmf-dist/tex/latex/base/ifthen.sty
Package: ifthen 2022/04/13 v1.1d Standard LaTeX ifthen package (DPC)
) (/usr/local/texlive/2024/texmf-dist/tex/latex/graphics/keyval.sty
Package: keyval 2022/05/29 v1.15 key=value parser (DPC)
\KV@toks@=\toks23
)
\c@ALC@unique=\count281
\c@ALC@line=\count282
\c@ALC@rem=\count283
\c@ALC@depth=\count284
\ALC@tlm=\skip54
\algorithmicindent=\skip55
) (/usr/local/texlive/2024/texmf-dist/tex/latex/graphics/graphicx.sty
Package: graphicx 2021/09/16 v1.2d Enhanced LaTeX Graphics (DPC,SPQR)
(/usr/local/texlive/2024/texmf-dist/tex/latex/graphics/graphics.sty
Package: graphics 2022/03/10 v1.4e Standard LaTeX Graphics (DPC,SPQR)
(/usr/local/texlive/2024/texmf-dist/tex/latex/graphics/trig.sty
Package: trig 2021/08/11 v1.11 sin cos tan (DPC)
) (/usr/local/texlive/2024/texmf-dist/tex/latex/graphics-cfg/graphics.cfg
File: graphics.cfg 2016/06/04 v1.11 sample graphics configuration
)
Package graphics Info: Driver file: pdftex.def on input line 107.
(/usr/local/texlive/2024/texmf-dist/tex/latex/graphics-def/pdftex.def
File: pdftex.def 2022/09/22 v1.2b Graphics/color driver for pdftex
))
\Gin@req@height=\dimen173
\Gin@req@width=\dimen174
) (/usr/local/texlive/2024/texmf-dist/tex/latex/base/textcomp.sty
Package: textcomp 2020/02/02 v2.0n Standard LaTeX package
) (/usr/local/texlive/2024/texmf-dist/tex/latex/xcolor/xcolor.sty
Package: xcolor 2023/11/15 v3.01 LaTeX color extensions (UK)
(/usr/local/texlive/2024/texmf-dist/tex/latex/graphics-cfg/color.cfg
File: color.cfg 2016/01/02 v1.6 sample color configuration
)
Package xcolor Info: Driver file: pdftex.def on input line 274.
(/usr/local/texlive/2024/texmf-dist/tex/latex/graphics/mathcolor.ltx)
Package xcolor Info: Model `cmy' substituted by `cmy0' on input line 1350.
Package xcolor Info: Model `hsb' substituted by `rgb' on input line 1354.
Package xcolor Info: Model `RGB' extended on input line 1366.
Package xcolor Info: Model `HTML' substituted by `rgb' on input line 1368.
Package xcolor Info: Model `Hsb' substituted by `hsb' on input line 1369.
Package xcolor Info: Model `tHsb' substituted by `hsb' on input line 1370.
Package xcolor Info: Model `HSB' substituted by `hsb' on input line 1371.
Package xcolor Info: Model `Gray' substituted by `gray' on input line 1372.
Package xcolor Info: Model `wave' substituted by `hsb' on input line 1373.
) (/usr/local/texlive/2024/texmf-dist/tex/latex/l3backend/l3backend-pdftex.def
File: l3backend-pdftex.def 2024-03-14 L3 backend support: PDF output (pdfTeX)
\l__color_backend_stack_int=\count285
\l__pdf_internal_box=\box55
) (./output.aux)
\openout1 = `output.aux'.


LaTeX Font Info:    Checking defaults for OML/cmm/m/it on input line 12.
LaTeX Font Info:    ... okay on input line 12.
LaTeX Font Info:    Checking defaults for OMS/cmsy/m/n on input line 12.
LaTeX Font Info:    ... okay on input line 12.
LaTeX Font Info:    Checking defaults for OT1/cmr/m/n on input line 12.
LaTeX Font Info:    ... okay on input line 12.
LaTeX Font Info:    Checking defaults for T1/cmr/m/n on input line 12.
LaTeX Font Info:    ... okay on input line 12.
LaTeX Font Info:    Checking defaults for TS1/cmr/m/n on input line 12.
LaTeX Font Info:    ... okay on input line 12.
LaTeX Font Info:    Checking defaults for OMX/cmex/m/n on input line 12.
LaTeX Font Info:    ... okay on input line 12.
LaTeX Font Info:    Checking defaults for U/cmr/m/n on input line 12.
LaTeX Font Info:    ... okay on input line 12.
-- Lines per column: 56 (exact).
(/usr/local/texlive/2024/texmf-dist/tex/context/base/mkii/supp-pdf.mkii
[Loading MPS to PDF converter (version 2006.09.02).]
\scratchcounter=\count286
\scratchdimen=\dimen175
\scratchbox=\box56
\nofMPsegments=\count287
\nofMParguments=\count288
\everyMPshowfont=\toks24
\MPscratchCnt=\count289
\MPscratchDim=\dimen176
\MPnumerator=\count290
\makeMPintoPDFobject=\count291
\everyMPtoPDFconversion=\toks25
) (/usr/local/texlive/2024/texmf-dist/tex/latex/epstopdf-pkg/epstopdf-base.sty
Package: epstopdf-base 2020-01-24 v2.11 Base part for package epstopdf
Package epstopdf-base Info: Redefining graphics rule for `.eps' on input line 485.
(/usr/local/texlive/2024/texmf-dist/tex/latex/latexconfig/epstopdf-sys.cfg
File: epstopdf-sys.cfg 2010/07/13 v1.3 Configuration of (r)epstopdf for TeX Live
))
LaTeX Font Info:    Trying to load font information for U+msa on input line 27.
(/usr/local/texlive/2024/texmf-dist/tex/latex/amsfonts/umsa.fd
File: umsa.fd 2013/01/14 v3.01 AMS symbols A
)
LaTeX Font Info:    Trying to load font information for U+msb on input line 27.
(/usr/local/texlive/2024/texmf-dist/tex/latex/amsfonts/umsb.fd
File: umsb.fd 2013/01/14 v3.01 AMS symbols B
)
<fig1.JPG, id=1, 237.88875pt x 148.30406pt>
File: fig1.JPG Graphic file (type jpg)
<use fig1.JPG>
Package pdftex.def Info: fig1.JPG  used on input line 42.
(pdftex.def)             Requested size: 226.79846pt x 141.3972pt.
[1{/usr/local/texlive/2024/texmf-var/fonts/map/pdftex/updmap/pdftex.map}{/usr/local/texlive/2024/texmf-dist/fonts/enc/dvips/base/8r.enc}




 <./fig1.JPG>]
<fgi2.JPG, id=13, 289.08pt x 249.93375pt>
File: fgi2.JPG Graphic file (type jpg)
<use fgi2.JPG>
Package pdftex.def Info: fgi2.JPG  used on input line 67.
(pdftex.def)             Requested size: 252.0pt x 217.87537pt.


Overfull \hbox (10.0pt too wide) in paragraph at lines 67--68
[][] 
 []


<Screenshot 2024-11-30 at 7.07.04 PM.png, id=14, 246.9225pt x 224.84pt>
File: Screenshot 2024-11-30 at 7.07.04 PM.png Graphic file (type png)
<use Screenshot 2024-11-30 at 7.07.04 PM.png>
Package pdftex.def Info: Screenshot 2024-11-30 at 7.07.04 PM.png  used on input line 75.
(pdftex.def)             Requested size: 201.60077pt x 183.57062pt.
<Screenshot 2024-11-30 at 7.00.34 PM.png, id=16, 246.9225pt x 220.825pt>
File: Screenshot 2024-11-30 at 7.00.34 PM.png Graphic file (type png)
<use Screenshot 2024-11-30 at 7.00.34 PM.png>
Package pdftex.def Info: Screenshot 2024-11-30 at 7.00.34 PM.png  used on input line 81.
(pdftex.def)             Requested size: 201.60077pt x 180.29256pt.
<figs3thru5fix.JPG, id=17, 447.17062pt x 465.99094pt>
File: figs3thru5fix.JPG Graphic file (type jpg)
<use figs3thru5fix.JPG>
Package pdftex.def Info: figs3thru5fix.JPG  used on input line 88.
(pdftex.def)             Requested size: 201.60077pt x 210.08505pt.
<table1.JPG, id=18, 247.67531pt x 79.04532pt>
File: table1.JPG Graphic file (type jpg)
<use table1.JPG>
Package pdftex.def Info: table1.JPG  used on input line 96.
(pdftex.def)             Requested size: 201.60077pt x 64.34114pt.
<fig6.JPG, id=19, 513.41812pt x 351.56343pt>
File: fig6.JPG Graphic file (type jpg)
<use fig6.JPG>
Package pdftex.def Info: fig6.JPG  used on input line 99.
(pdftex.def)             Requested size: 252.0pt x 172.55725pt.


Overfull \hbox (10.0pt too wide) in paragraph at lines 99--101
[][]
 []




Underfull \hbox (badness 1442) in paragraph at lines 99--101
\OT1/ptm/m/n/10 The elec-tric field for this salt-wa-ter can be de-fined by
 []


[2 <./fgi2.JPG> <./Screenshot 2024-11-30 at 7.07.04 PM.png> <./Screenshot 2024-11-30 at 7.00.34 PM.png> <./figs3thru5fix.JPG> <./table1.JPG> <./fig6.JPG>]
<fig7.JPG, id=30, 424.58624pt x 488.57532pt>
File: fig7.JPG Graphic file (type jpg)
<use fig7.JPG>
Package pdftex.def Info: fig7.JPG  used on input line 108.
(pdftex.def)             Requested size: 226.79846pt x 260.97891pt.


Underfull \hbox (badness 10000) in paragraph at lines 108--110
[][]
 []




Underfull \hbox (badness 10000) in paragraph at lines 132--136
[]\OT1/ptm/m/n/10 [1] "7.6: Equipo-ten-tial Sur-faces and Con-duc-tors."
 []




Underfull \hbox (badness 10000) in paragraph at lines 132--136
\OT1/ptm/m/n/10 *Physics Li-bre-Texts*, Li-bre-Texts, 1 Oct. 2024,
 []




Underfull \hbox (badness 10000) in paragraph at lines 132--136
\OT1/ptm/m/n/10 Physics 28Open-Stax29/ Uni-ver-sity Physics II -
 []




** Conference Paper **
Before submitting the final camera ready copy, remember to:


 1. Manually equalize the lengths of two columns on the last page
 of your paper;


 2. Ensure that any PostScript and/or PDF output post-processing
 uses only Type 1 fonts and that every step in the generation
 process uses the appropriate paper size.


[3 <./fig7.JPG>] (./output.aux)
 ***********
LaTeX2e <2023-11-01> patch level 1
L3 programming layer <2024-03-14>
 ***********
 ) 
Here is how much of TeX's memory you used:
 4182 strings out of 474104
 65757 string characters out of 5743486
 1939493 words of memory out of 5000000
 26530 multiletter control sequences out of 15000+600000
 587999 words of font info for 91 fonts, out of 8000000 for 9000
 1141 hyphenation exceptions out of 8191
 57i,8n,65p,1432b,255s stack positions out of 10000i,1000n,20000p,200000b,200000s
</usr/local/texlive/2024/texmf-dist/fonts/type1/public/amsfonts/cm/cmex10.pfb></usr/local/texlive/2024/texmf-dist/fonts/type1/public/amsfonts/cm/cmmi10.pfb></usr/local/texlive/2024/texmf-dist/fonts/type1/public/amsfonts/cm/cmmi7.pfb></usr/local/texlive/2024/texmf-dist/fonts/type1/public/amsfonts/cm/cmr10.pfb></usr/local/texlive/2024/texmf-dist/fonts/type1/public/amsfonts/cm/cmr5.pfb></usr/local/texlive/2024/texmf-dist/fonts/type1/public/amsfonts/cm/cmr7.pfb></usr/local/texlive/2024/texmf-dist/fonts/type1/public/amsfonts/cm/cmsy10.pfb></usr/local/texlive/2024/texmf-dist/fonts/type1/public/amsfonts/cm/cmsy7.pfb></usr/local/texlive/2024/texmf-dist/fonts/type1/urw/times/utmb8a.pfb></usr/local/texlive/2024/texmf-dist/fonts/type1/urw/times/utmbi8a.pfb></usr/local/texlive/2024/texmf-dist/fonts/type1/urw/times/utmr8a.pfb>
Output written on output.pdf (3 pages, 356397 bytes).
PDF statistics:
 81 PDF objects out of 1000 (max. 8388607)
 43 compressed objects within 1 object stream
 0 named destinations out of 1000 (max. 500000)
 41 words of extra memory for PDF output out of 10000 (max. 10000000)